\documentclass[11pt]{article}

\usepackage[margin=1.0in]{geometry}
\usepackage{amsmath}
\usepackage[
	colorlinks=true,
]{hyperref}

% Brackets
\newcommand{\brackets}[1]{\left(#1\right)}
\newcommand{\sqbrackets}[1]{\left[#1\right]}
\newcommand{\cbrackets}[1]{\left\{#1\right\}}

%  Vectors
\renewcommand{\vec}[1]{\boldsymbol#1}
\newcommand{\unitv}[1]{\hat{\vec{#1}}}

% Basis vectors
\newcommand{\co}[1]{\vec{e}_{#1}}
\newcommand{\con}[1]{\vec{e}^{#1}}

\newcommand{\pd}[2]{\frac{\partial#1}{\partial#2}}

% Equations
\newcommand{\equ}[2]{
	\begin{equation}
    \begin{split}
	#1
	\label{#2}
	\end{split}
	\end{equation}
}

\newcommand{\dotp}[2]{#1\cdot#2}
\newcommand{\crossp}[2]{#1\times#2}

% Operators
\newcommand{\curl}[1]{\crossp{\nabla}{#1}}
\newcommand{\divergence}[1]{\dotp{\nabla}{#1}}

\title{SIESTA Boundary Condition Notes}
\author{Mark Cianciosa}
\date{\today}

\begin{document}

\maketitle

\tableofcontents

\section{Introduction}
Handling the boundaries in SIESTA and by extension VMEC is critical to the correctness of the solution. 
This document notes derives the boundary conditions for real and Fourier quantities for various operations and quantities in VMEC.

\section{Basis Vectors}
The SIESTA basis vectors are defined by the geometry $R$ and $Z$ quantities. 
For the stellarator symmetric case, $R$ does not swap sign across the origin while $Z$ does except at the center.
This implies a cosine and sine parities of these quantities respectively.
See section \ref{sub:sec:real_space} for the conditions on boundary conditions at the origin.
At the origin, the $R$ and $Z$ values do not change with the poloidal angle.
This implies that all $m>0$ terms must be zero.

\subsection{Derivatives of $R$ and $Z$}
$R$ at $u=\frac{\pi}{2},v=0$ and $u=-\frac{\pi}{2},v=0$ is symmetric above and below the origin.
This mirror image implies that radial derivatives must have opposite signs with a zero derivative at the origin.
However at $u=0,v=0$ and $u=\pi,v=0$, the values changes with out a sign flip implying a non zero derivative.
Similarly, $Z$ at since $Z$ is zero at $u=0,v=0$ and $u=\pi,v=0$, the origin derivative should be zero as well.
At other angles, the value of $Z$ changes sign.
Both of these factors imply a $m=1$ poloidal mode structure to any radial derivative quantity.
\equ{
\pd{A\brackets{s=0}}{s}=
\begin{array}{lr}
0 & m\ne1
\end{array}
}{equ:dads}

From section \ref{sub:sec:real_space} $R$ and $Z$ at the origin, lack any $m>0$ terms.
This causes all poloidal derivatives to be zero.
\equ{
\pd{A\brackets{s=0}}{u}=0
}{equ:dadu}
This is consistent with the view that origin is a single point so any move the $u$ direction results in no change in the $R,Z$ position.
Note this condition is not universal as we will see for vector quantities.
In the toroidal direction, there are no special conditions.

\section{Basis Vector}
The covariant basis vectors are defined by derivatives with respect to the coordinates.
\equ{
\co{i}=\pd{R\unitv{r}}{i}+\pd{Z\unitv{z}}{i}
}{equ:basis}
The $\co{s}$ basis vector becomes,
\equ{
\co{s}=\pd{R}{s}\unitv{r}+\pd{Z}{s}\unitv{z}
}{equ:esubs}
Since, radial derivatives only have $m=1$ terms, 
\equ{
A_{s}=\dotp{\vec{A}}{\co{s}}
}{equ:asubs}
the contravariant components only contain $m=1$ terms at $s=0$.
Note this implies a non-zero poloidal derivative.

From equation \ref{equ:dadu}, the $\co{u}$ basis should be zero at $s=0$.
\equ{
\co{u}=\pd{R}{u}\unitv{r}+\pd{Z}{u}\unitv{z}=0
}{equ:esubu}
This implies that at the origin 
\equ{
A_{u}=\dotp{\vec{A}}{\co{u}}=0
}{equ:asubu}

Since toroidal derivatives retain the m=0 mode structure,
\equ{
\co{v}=\pd{R}{v}\unitv{r}+R\unitv{\phi}+\pd{Z}{v}\unitv{z}
}{equ:esubv}
non-zero only for $m=0$ terms.
Since $\pd{R}{v}$, $\pd{Z}{v}$, and $R$ are zero for any $m>0$ this implies that
\equ{
A_{v}=\dotp{\vec{A}}{\co{v}}
}{equ:asubv}
the contravariant components only contain $m=0$ terms at $s=0$.

\subsection{Jacobian}
The Jacobian of the this coordinate system is obtained from
\equ{
\jmath=\dotp{\co{s}}{\crossp{\co{u}}{\co{v}}}
}{equ:jacobian}
Because of condition \ref{equ:esubu}, the Jacobian should be zero at the origin.
This implies that quantities times the Jacobian would be zero.
\equ{
\begin{array}{l}
\jmath A^{i}=0 \\
\jmath A_{i}=0
\end{array}
}{equ:jacobian_vector}
This futher implies that poloidal and toroidal derivatives at the origin are zero as well.
\equ{
\begin{array}{l}
\pd{\jmath A^{i}}{u}=0 \\
\pd{\jmath A^{i}}{v}=0
\end{array}
}{equ:jacobian_dervative}
However we will see that this is not necessarily the case for $\mathcal{J}A^{u}$ $m=1$ components.

\section{Physics Quantities}
Some of the Fourier modes represent physical quantities of interest.

\subsection{Magnetic Flux}
The magnetic flux through a surface is defined as 
\equ{
\Phi=\iint\dotp{\vec{B}}{d\vec{a}}=\oint\dotp{\vec{A}}{d\vec{l}}
}{equ:magnetic_flux}
Poloidal flux is defined to be
\equ{
\chi=\oint\dotp{\vec{A}}{\co{v}dv}=\oint A_{v}dv=2\pi A^{m0}_{v}\cos\brackets{mu}
}{equ:poloidal_flux}
While in general, the polodial flux does not need to be zero at the origin, we can choose that point as the zero reference.
From the condition \ref{equ:covariant_boundary_v} at the origin, $m>0$ terms of $A_{v}$ are zero implying that $\chi=2\pi A^{00}_{v}=0$.

Alternatively, using the area integration,
\equ{
\chi=\iint\dotp{\vec{B}}{\mathcal{J}\con{u}dvds}=2\pi\cos\brackets{mu}\int\mathcal{J}B_{m0}^{u}ds
}{equ:poloidal_flux_area}
This implies 
\equ{
\chi'=2\pi\mathcal{J}B_{m0}^{u}\cos\brackets{mu}
}{equ:poloidal_flux_prime}
At the origin from equation \ref{equ:jacobian_vector}
\equ{
\chi'=2\pi\mathcal{J}B_{m0}^{u}\cos\brackets{mu}=0
}{equ:poloidal_flux_prime_origin}

Similarly for the toroidal flux, 
\equ{
\phi=\oint\dotp{\vec{A}}{\co{u}du}=\oint A_{u}du=2\pi A^{0n}_{u}\cos\brackets{nNv}
}{equ:toroidal_flux}
At the origin, the toroidal area goes to zero implying that toroidal flux goes to zero.
From equation \ref{equ:asubu}, $A_u=0$ for all values of $n$ consistent with the condition of the zero area.

Alternatively, using the area integration,
\equ{
\phi=\iint\dotp{\vec{B}}{\mathcal{J}\con{v}duds}=2\pi\cos\brackets{nNv}\int\mathcal{J}B_{0n}^{v}ds
}{equ:toroidal_flux_area}
This implies that
\equ{
\phi'=2\pi\mathcal{J}B_{0n}^{v}\cos\brackets{nNv}
}{equ:toroidal_flux_prime}
At the origin from equation \ref{equ:jacobian_vector},
\equ{
\phi'=2\pi\mathcal{J}B_{0n}^{v}\cos\brackets{nNv}=0
}{equ:toroidal_flux_prime_origin}

\subsection{Toroidal Current}
From Ampere's law, the total torodial current is 
\equ{
\mu_{0}I=\mu_{0}\iint\dotp{\vec{J}}{d\vec{a}}=\oint\dotp{\vec{B}}{d\vec{l}}=\oint\dotp{\vec{B}}{e_{u}du}=\oint B_{u}du=2\pi B^{0n}_{u}\cos{nNv}
}{equ:amperes_law}
Since the integration area is zero at the origin $B^{0n}_{u}\brackets{s=0}=0$.
This is consistent with the condition \ref{equ:asubu}.
At the edge, the total current is enclosed so $2\pi B^{0n}_{u}\brackets{s=1}\cos{nNv}=\mu_{0}I_{Tor}$.
This implies that $B^{0n}_{u}\brackets{s=1}=0$ for $n>0$.

\section{Fourier Quantities}
VMEC and SIESTA use a Fourier representation quantities.
Stellarator symmetric equilibrium quantities contain either a sine or cosine parity.
Non-stellarator symmetric equilibrium quantities contain both parities.

\subsection{Conversion To Real Space}
\label{sub:sec:real_space}
SIESTA quantities can be converted to real space by summing over all Fourier modes.
\equ{
A_{real}\brackets{s,u,v}=\sum^{mpol}_{m=0}\sum^{ntor}_{n=-ntor} A_{nmc}\brackets{s}\cos\brackets{mu+nNv} + A_{nms}\brackets{s}\sin\brackets{mu+nNv}
}{equ:fourier_to_real}
$s$ is a the radial coordinate, $u$ is the poloidal angle, and $v$ is the toroidal angle.
$m$ is the poloidal mode number, $n$ is the toroidal mode number, and $N$ is the number of field periods.
$mpol$ is the total number of poloidal modes and $ntor$ is the total number of polidal modes.
$A_{nmc}$ are the Fourier amplitudes for the cosine parity and $A_{nms}$ are the Fourier amplitudes for the sine parity.
For the $m=0$ modes, all $n<0$ amplitudes are zero.
For sin parities, the $m=0,n=0$ amplitude is zero as well since
\equ{
\sin\brackets{0u+0Nv}=0.
}{equ:fourier_sine_00}
Radial quantities are defined on their a full or half grid.
Typically base quantities are defined on the full grid.
Derivatives of full grid quantities are defined on the half grid and derivatives of half grid quantities are defined on the full grid.

\subsection{Cross Products}
In curvalinear coordinates, the cross product of contravariant components is defined as
\equ{
\crossp{\vec{A}}{\vec{B}}=\sum\brackets{\mathcal{J}A^{i}B^{j}-\mathcal{J}A^{j}B^{i}}\co{k}
}{equ:cross_con}
for cyclic permutations of $i,j,k$.
Typical SIESTA quantities have sine parity in the $s$ components and cosine parity in the $u,v$ components.
Note since this doesn't involve any derivatives, we do not need to be concerned about convervsions between half and full grids.
This operation only involves terms of constant s.

\subsubsection{Sine $s$ Parity}
Starting with the $e_{s}$ component the cross product can be written as.
\equ{
\brackets{\crossp{\vec{A}}{\vec{B}}}^{s}=
\mathcal{J}A^{u}\cos\brackets{m_{A}u+n_{A}v}B^{v}\cos\brackets{m_{B}u+n_{B}v}\\
-\mathcal{J}A^{v}\cos\brackets{m_{A}u+n_{A}v}B^{u}\cos\brackets{m_{B}u+n_{B}v}
}{equ:cross_esubs_sine}
From the sum to product rules, this operation reduces to.
\equ{
\brackets{\crossp{\vec{A}}{\vec{B}}}^{s}=
\frac{1}{2}\mathcal{J}A^{u}B^{v}\cos\sqbrackets{\brackets{m_{A}-m_{B}}u+\brackets{n_{A}-n_{B}}v}\\
+\frac{1}{2}\mathcal{J}A^{u}B^{v}\cos\sqbrackets{\brackets{m_{A}+m_{B}}u+\brackets{n_{A}+n_{B}}v}\\
-\frac{1}{2}\mathcal{J}A^{v}B^{u}\cos\sqbrackets{\brackets{m_{A}-m_{B}}u+\brackets{n_{A}-n_{B}}v}\\
-\frac{1}{2}\mathcal{J}A^{v}B^{u}\cos\sqbrackets{\brackets{m_{A}+m_{B}}u+\brackets{n_{A}+n_{B}}v}
}{equ:cross_esubs2_sine}

For the $e_{u}$ component the cross product can be written as.
\equ{
\brackets{\crossp{\vec{A}}{\vec{B}}}^{u}=
\mathcal{J}A^{v}\cos\brackets{m_{A}u+n_{A}v}B^{s}\sin\brackets{m_{B}u+n_{B}v}\\
-\mathcal{J}A^{s}\sin\brackets{m_{A}u+n_{A}v}B^{v}\cos\brackets{m_{B}u+n_{B}v}
}{equ:cross_esubu_sine}
From the sum to product rules, this operation reduces to.
\equ{
\brackets{\crossp{\vec{A}}{\vec{B}}}^{u}=
\frac{1}{2}\mathcal{J}A^{v}B^{s}\sin\sqbrackets{\brackets{m_{A}+m_{B}}u+\brackets{n_{A}+n_{B}}v}\\
-\frac{1}{2}\mathcal{J}A^{v}B^{s}\sin\sqbrackets{\brackets{m_{A}-m_{B}}u+\brackets{n_{A}-n_{B}}v}\\
-\frac{1}{2}\mathcal{J}A^{s}B^{v}\sin\sqbrackets{\brackets{m_{A}+m_{B}}u+\brackets{n_{A}+n_{B}}v}\\
-\frac{1}{2}\mathcal{J}A^{s}B^{v}\sin\sqbrackets{\brackets{m_{A}-m_{B}}u+\brackets{n_{A}-n_{B}}v}
}{equ:cross_esubu2_sine}

For the $e_{v}$ component the cross product can be written as.
\equ{
\brackets{\crossp{\vec{A}}{\vec{B}}}^{v}=
\mathcal{J}A^{s}\sin\brackets{m_{A}u+n_{A}v}B^{v}\cos\brackets{m_{B}u+n_{B}v}\\
-\mathcal{J}A^{u}\cos\brackets{m_{A}u+n_{A}v}B^{s}\sin\brackets{m_{B}u+n_{B}v}
}{equ:cross_esubv_sine}
From the sum to product rules, this operation reduces to.
\equ{
\brackets{\crossp{\vec{A}}{\vec{B}}}^{v}=
\frac{1}{2}\mathcal{J}A^{s}B^{u}\sin\sqbrackets{\brackets{m_{A}+m_{B}}u+\brackets{n_{A}+n_{B}}v}\\
+\frac{1}{2}\mathcal{J}A^{s}B^{u}\sin\sqbrackets{\brackets{m_{A}-m_{B}}u+\brackets{n_{A}-n_{B}}v}\\
-\frac{1}{2}\mathcal{J}A^{u}B^{s}\sin\sqbrackets{\brackets{m_{A}+m_{B}}u+\brackets{n_{A}+n_{B}}v}\\
+\frac{1}{2}\mathcal{J}A^{u}B^{s}\sin\sqbrackets{\brackets{m_{A}-m_{B}}u+\brackets{n_{A}-n_{B}}v}
}{equ:cross_esubv2_sine}
For this operation, the resultant quantity flips parity.

\subsubsection{Cosine $s$ Parity}
Starting with the $e_{s}$ component the cross product can be written as.
\equ{
\brackets{\crossp{\vec{A}}{\vec{B}}}^{s}=
\mathcal{J}A^{u}\sin\brackets{m_{A}u+n_{A}v}B^{v}\sin\brackets{m_{B}u+n_{B}v}\\
-\mathcal{J}A^{v}\sin\brackets{m_{A}u+n_{A}v}B^{u}\sin\brackets{m_{B}u+n_{B}v}
}{equ:cross_esubs_cosine}
From the sum to product rules, this operation reduces to.
\equ{
\brackets{\crossp{\vec{A}}{\vec{B}}}^{s}=
\frac{1}{2}\mathcal{J}A^{u}B^{v}\cos\sqbrackets{\brackets{m_{A}-m_{B}}u+\brackets{n_{A}-n_{B}}v}\\
-\frac{1}{2}\mathcal{J}A^{u}B^{v}\cos\sqbrackets{\brackets{m_{A}+m_{B}}u+\brackets{n_{A}+n_{B}}v}\\
-\frac{1}{2}\mathcal{J}A^{v}B^{u}\cos\sqbrackets{\brackets{m_{A}-m_{B}}u+\brackets{n_{A}-n_{B}}v}\\
+\frac{1}{2}\mathcal{J}A^{v}B^{u}\cos\sqbrackets{\brackets{m_{A}+m_{B}}u+\brackets{n_{A}+n_{B}}v}
}{equ:cross_esubs2_cosine}

For the $e_{u}$ component the cross product can be written as.
\equ{
\brackets{\crossp{\vec{A}}{\vec{B}}}^{u}=
\mathcal{J}A^{v}\sin\brackets{m_{A}u+n_{A}v}B^{s}\cos\brackets{m_{B}u+n_{B}v}\\
-\mathcal{J}A^{s}\cos\brackets{m_{A}u+n_{A}v}B^{v}\sin\brackets{m_{B}u+n_{B}v}
}{equ:cross_esubu_cosine}
From the sum to product rules, this operation reduces to.
\equ{
\brackets{\crossp{\vec{A}}{\vec{B}}}^{u}=
\frac{1}{2}\mathcal{J}A^{v}B^{s}\sin\sqbrackets{\brackets{m_{A}+m_{B}}u+\brackets{n_{A}+n_{B}}v}\\
-\frac{1}{2}\mathcal{J}A^{v}B^{s}\sin\sqbrackets{\brackets{m_{A}-m_{B}}u+\brackets{n_{A}-n_{B}}v}\\
-\frac{1}{2}\mathcal{J}A^{s}B^{v}\sin\sqbrackets{\brackets{m_{A}+m_{B}}u+\brackets{n_{A}+n_{B}}v}\\
-\frac{1}{2}\mathcal{J}A^{s}B^{v}\sin\sqbrackets{\brackets{m_{A}-m_{B}}u+\brackets{n_{A}-n_{B}}v}
}{equ:cross_esubu2_cosine}

For the $e_{v}$ component the cross product can be written as.
\equ{
\brackets{\crossp{\vec{A}}{\vec{B}}}^{v}=\mathcal{J}A^{s}\sin\brackets{m_{A}u+n_{A}v}B^{v}\cos\brackets{m_{B}u+n_{B}v}\\
-\mathcal{J}A^{u}\cos\brackets{m_{A}u+n_{A}v}B^{s}\sin\brackets{m_{B}u+n_{B}v}
}{equ:cross_esubv_cosine}
From the sum to product rules, this operation reduces to.
\equ{
\brackets{\crossp{\vec{A}}{\vec{B}}}^{v}=
\frac{1}{2}\mathcal{J}A^{s}B^{u}\sin\sqbrackets{\brackets{m_{A}+m_{B}}u+\brackets{n_{A}+n_{B}}v}\\
+\frac{1}{2}\mathcal{J}A^{s}B^{u}\sin\sqbrackets{\brackets{m_{A}-m_{B}}u+\brackets{n_{A}-n_{B}}v}\\
-\frac{1}{2}\mathcal{J}A^{u}B^{s}\sin\sqbrackets{\brackets{m_{A}+m_{B}}u+\brackets{n_{A}+n_{B}}v}\\
+\frac{1}{2}\mathcal{J}A^{u}B^{s}\sin\sqbrackets{\brackets{m_{A}-m_{B}}u+\brackets{n_{A}-n_{B}}v}
}{equ:cross_esubv2_cosine}
For this operation, the resultant quantity flips parity.

\subsection{Boundary Conditions}
At the center of the computation grid, the position does not vary with changes in the $u$ coordinate.
This implies that the $\co{u}\brackets{s=0}=0$.
As a consequence, $A_{u}\brackets{s=0}=\vec{A}\cdot\co{s=0}=0$.
Additionally in the limit of the grid axis, the grid becomes cylindrical thus the $\con{u}=0$ implying that $A^{u}\brackets{s=0}=\vec{A}\cdot\con{s=0}=0$.
However we need to consider these in the limit as $s\rightarrow 0$.
Radial basis vectors change direction based on $u$ so radial $s$ components should have $m=1$ components only.
The toroidal direction doesn't change as $u$ varies so toroidal components should only have $m=0$ components only. The full boundary conditions are.

\subsubsection{Covariant}
\equ{
A^{full}_{s}\brackets{s=0}=0\quad m \ne 1
}{equ:covariant_boundary_s}
\equ{
A^{full}_{u}\brackets{s=0}=0
}{equ:covariant_boundary_u}
\equ{
A^{full}_{v}\brackets{s=0}=0\quad m \ne 0
}{equ:covariant_boundary_v}

\subsubsection{Contravariant}
Since the Jacobian, equation \ref{equ:jacobian}, is zero at the origin due to equation \ref{equ:covariant_boundary_u}, we cannot come up with a boundary condition for the contravariant basis vectors directly.
However we can define the Jaocbian times the contravariant basis
$\mathcal{J}\con{i}=\crossp{\co{j}}{\co{k}}$.
At the origin, all of these should be zero due to the Jacobian being zero. 
\equ{
\mathcal{J}\con{s}\brackets{s=0}=\crossp{\co{u}}{\co{v}}=0
}{equ:contravariant_boundary_s}
Since $\co{u}\brackets{s=0}=0$, this condition is satisfied.
\equ{
\mathcal{J}\con{u}\brackets{s=0}=\crossp{\co{v}}{\co{s}}
}{equ:contravariant_boundary_u}
From the definition of the cross product,
\equ{
\crossp{\co{v}}{\co{s}}=\left|
\begin{array}{ccc}
\unitv{r} & \unitv{\phi} & \unitv{z} \\
\pd{R}{v} & R            & \pd{Z}{v} \\
\pd{R}{s} & 0            & \pd{Z}{s}
\end{array}
\right|=R\pd{Z}{s}\unitv{r}+\brackets{\pd{Z}{v}\pd{R}{s}-\pd{R}{v}\pd{Z}{s}}\unitv{\phi}-R\pd{R}{s}\unitv{z}
}{equ:ev_cross_es}
At the origin, $R$ is only non-zero for $m=0$ terms and from equation \ref{equ:dads} radial derivative terms are only non-zero for $m=1$ terms.

For the $\unitv{r}$ component
\equ{
R_{m=0,n_{r}}\pd{Z}{s}_{m=1,n_{z}}\cos\brackets{n_{r}v}\sin\brackets{u+n_{z}v}
}{equ:rhat_ev_cross_es}
From the product to sum identity
\equ{
\frac{1}{2}R_{m=0,n_{r}}\pd{Z}{s}_{m=1,n_{z}}\sqbrackets{\sin\brackets{u+\brackets{n_{r}+n_{z}}v}+\sin\brackets{u+\brackets{n_{z}-n_{r}}v}}
}{equ:rhat_ev_cross_es_2}
From here we see that $\mathcal{J}\dotp{\con{u}}{\unitv{r}}$ has a non-zero $m=1$ mode structure.

For the $\unitv{\phi}$ component
\equ{
n_{z}Z_{m=0,n_{z}}\pd{R}{s}_{m=1,n_{r}}\cos\brackets{n_{z}v}\cos\brackets{u+n_{r}v}\\
+n_{r}R_{m=0,n_{r}}\pd{Z}{s}_{m=1,n_{z}}\sin\brackets{n_{r}v}\sin\brackets{u+n_{z}v}
}{equ:phihat_ev_cross_es}
From the product sum identity
\equ{
\frac{1}{2}n_{z}Z_{m=0,n_{z}}\pd{R}{s}_{m=1,n_{r}}\sqbrackets{\cos\brackets{u+\brackets{n_{r}-n_{z}}v}+\cos\brackets{u+\brackets{n_{r}+n_{z}}v}}\\
+\frac{1}{2}n_{r}R_{m=0,n_{r}}\pd{Z}{s}_{m=1,n_{z}}\sqbrackets{\cos\brackets{u+\brackets{n_{z}-n_{r}}v}-\cos\brackets{u+\brackets{n_{r}+n_{z}}v}}
}{equ:phihat_ev_cross_es_2}
From here we see that $\mathcal{J}\dotp{\con{u}}{\unitv{\phi}}$ has a non-zero $m=1$ mode structure.

For the $\unitv{z}$ component
\equ{
R_{m=0,n_{r}}\pd{R}{s}_{m=1,n_{z}}\cos\brackets{n_{r}v}\cos\brackets{u+n_{z}v}
}{equ:zhat_ev_cross_es}
From the product sum identity
\equ{
\frac{1}{2}R_{m=0,n_{r}}\pd{R}{s}_{m=1,n_{z}}\sqbrackets{\cos\brackets{u+\brackets{n_{z}-n_{r}}v}-\cos\brackets{u+\brackets{n_{r}+n_{z}}v}}
}{equ:zhat_ev_cross_es_2}
From here we see that $\mathcal{J}\dotp{\con{u}}{\unitv{z}}$ has a non-zero $m=1$ mode structure.

\equ{
\mathcal{J}\con{v}\brackets{s=0}=\crossp{\co{s}}{\co{u}}=0
}{equ:covtravariant_boundary_v}
Since $\co{u}\brackets{s=0}=0$, this condition is satisfied.

\equ{
\mathcal{J}A_{full}^{s}\brackets{s=0}=0
}{equ:covariant_boundary_s}
\equ{
\mathcal{J}A_{full}^{u}\brackets{s=0}=0\quad m \ne 1
}{equ:covariant_boundary_u}
\equ{
\mathcal{J}A_{full}^{v}\brackets{s=0}=0
}{equ:covariant_boundary_v}

\subsection{Conversion Between Full and Half Gird}
Sometimes it is necessary to obtain a full grid quantity on the half grid or vice versa.
Full grid quantities are typically defined from index $1$ to $ns$.
Half grid quantities are defined from $2$ to $ns$.
For half grid quantities, the $1$ index is typically zeroed out or used to store other data.

\subsubsection{Conversion From Full to Half Grid}
Full grid radial quantities are converted to half grid by averaging.
\equ{
A^{i}_{half}=\frac{A^{i}_{full}+A^{i-1}_{full}}{2}\quad i=2\rightarrow ns
}{equ:full_to_half}
Since the resulting half grid quantity remains within the bounds, there are no special boundary conditions.

\subsubsection{Conversion From Half to Full Grid}
For indices that do not intersect the boundary, half grid quantities are converted to full grid by averaging.
\equ{
A^{i}_{full}=\frac{A^{i+1}_{half}+A^{i}_{half}}{2}\quad i=2\rightarrow ns-1
}{equ:half_to_full_middle}
The $ns$ index must be extrapolated in away to preserve the diveregence free condition.
Extrapolating the $ns$ point from the last half grid point yields.
\equ{
A^{ns}_{full}=\frac{3}{2}A^{ns}_{half}-\frac{1}{2}A^{ns-1}_{half}
}{equ:half_to_full_ns}

For $1$ full index, we can average between the $2$ half and $-2$ ghost half index.
The ghost point $-2$ should be the same as the no ghost point $2$ evaluated at a $u+\pi$ angle.
\equ{
A^{-2}_{half}\brackets{s,u,v}=A^{2}_{half}\brackets{s,u+\pi,v}
}{equ:half_ghost}
From equation \ref{equ:fourier_to_real} the ghost points become
\equ{
A^{-2}_{half}\brackets{s,u,v}=\sum^{mpol}_{m=0}\sum^{ntor}_{n=-ntor} A_{nmc}\brackets{s}\cos\brackets{m\brackets{u+\pi}+nNv} + A_{nms}\brackets{s}\sin\brackets{m\brackets{u+\pi}+nNv}
}{equ:half_ghost_2}
\equ{
A^{-2}_{half}\brackets{s,u,v}=\sum^{mpol}_{m=0}\sum^{ntor}_{n=-ntor} A_{nmc}\brackets{s}\cos\brackets{mu+nNv+m\pi} + A_{nms}\brackets{s}\sin\brackets{mu+nNv+m\pi}
}{equ:half_ghost_3}
\equ{
\begin{split}
A^{-2}_{half}\brackets{s,u,v}=\sum^{mpol}_{m=0}\sum^{ntor}_{n=-ntor} A_{nmc}\brackets{s}\sqbrackets{\cos\brackets{mu+nNv}\cos\brackets{m\pi} -\sin\brackets{mu+nNv}\sin\brackets{m\pi}}\\+ A_{nms}\brackets{s}\sqbrackets{\sin\brackets{mu+nNv}\cos\brackets{m\pi}+\cos\brackets{mu+nNv}\sin\brackets{m\pi}}
\end{split}
}{equ:half_ghost_4}
$\sin\brackets{m\pi}$ is zero for any value of m.
Equation 4 reduces to
\equ{
A^{-2}_{half}\brackets{s,u,v}=\sum^{mpol}_{m=0}\sum^{ntor}_{n=-ntor} A_{nmc}\brackets{s}\cos\brackets{mu+nNv}\cos\brackets{m\pi}\\+ A_{nms}\brackets{s}\sin\brackets{mu+nNv}\cos\brackets{m\pi}
}{equ:half_ghost_5}
\equ{
\cos\brackets{m\pi}=\left\{
\begin{array}{ll}
1  & \textrm{: even} \quad m \\
-1 & \textrm{: odd } \quad m
\end{array}
\right.
}{equ:cos_pi}
Using equation \ref{equ:cos_pi} the ghost points correspond to the non-ghost point via
\equ{
A^{-2}_{half}=\left\{
\begin{array}{ll}
A^{2}_{half}  & \textrm{: even} \quad m \\
-A^{2}_{half} & \textrm{: odd } \quad m
\end{array}
\right.
}{half_ghost_relationship}
Using this condition, the odd $m$ modes will average to zero
\equ{
A^{1}_{full}=\left\{
\begin{array}{ll}
A^{2}_{half} & \textrm{:} \quad m=0 \\
0            & \textrm{:} \quad m\ne0
\end{array}
\right.
}{equ:half_to_full_1}
However in cases where the quantity contains a radial derivative, we must account for the change in direction of $s$.
For quantities composed of a radial derivative, the half to full grid conversion becomes,
\equ{
A^{1}_{full}=\left\{
\begin{array}{ll}
A^{2}_{half} & \textrm{:} \quad m=1 \\
0            & \textrm{:} \quad m\ne1
\end{array}
\right.
}{equ:half_to_full_2}

\subsubsection{Test conversion}
To check the conversion operators we expect that when we convert from one grid to other than back again, we expect to recover the same values back.
From the interpolation operation, the conversion back to the full grid requires no extrapolation from the $2$ to $ns-1$.
Putting equation \ref{equ:full_to_half} into equation \ref{equ:half_to_full_middle} yields
\equ{
A^{i}_{full}=\frac{A^{i+1}_{full}}{4}+\frac{A^{i}_{full}}{2}+\frac{A^{i-1}_{full}}{4}\quad i=2\rightarrow ns-1
}{equ:full_to_half_to_full_1}
This would indicate we should only convert quantities in one direction and never back.

\subsection{Derivatives}
Derivatives of Fourier quantities are handled differently depending on the changing coordinate.

\subsubsection{Angle Derivatives}
Angle derivatives can be done numerically in Fourier space and requires no change in grid.
From equation \ref{equ:fourier_to_real} derivatives with respect to the polodial angle are
\equ{
\frac{\partial}{\partial u}A_{real}\brackets{s,u,v}=\sum^{mpol}_{m=0}\sum^{ntor}_{n=-ntor} -mA_{nmc}\brackets{s}\sin\brackets{mu+nNv} + mA_{nms}\brackets{s}\cos\brackets{mu+nNv}
}{equ:fourier_du}
\equ{
\frac{\partial}{\partial v}A_{real}\brackets{s,u,v}=\sum^{mpol}_{m=0}\sum^{ntor}_{n=-ntor} -nNA_{nmc}\brackets{s}\sin\brackets{mu+nNv} + nNA_{nms}\brackets{s}\cos\brackets{mu+nNv}
}{equ:fourier_dv}

\subsubsection{Radial Derivative of Full Grid Quantity}
Taking a radial derivative in the full grid results in a half grid quantity.
\equ{
\frac{\partial A}{\partial s}^{i}_{half}=\frac{A^{i}_{full}-A^{i-1}_{full}}{ds}\quad i=2\rightarrow ns
}{equ:full_ds}
Radial derivatives of full grid have no boundary conditions.

\subsubsection{Radial Derivative of Half Grid Quantity}
Taking a radial derivative in the half grid results in a full grid quantity.
For non-boundary grid points, the radial derivative of a half grid quantity is
\equ{
\frac{\partial A}{\partial s}^{i}_{full}=\frac{A^{i+1}_{half}-A^{i}_{half}}{ds}\quad i=2\rightarrow ns-1
}{equ:half_ds}
The $ns$ index must be extrapolated.
Extrapolating the full grid $ns$ point must preserve $\nabla\cdot\nabla\times\vec{A}=0$
\equ{
\frac{\partial A}{\partial s}^{ns}_{full}=\frac{\partial A}{\partial s}^{ns-1}_{full}=\frac{A^{ns}_{half}-A^{ns-1}_{half}}{ds}
}{equ:half_ds_ns}
To preserve $\nabla\cdot\nabla\times\vec{A}=0$ the radial derivative at the axis must take the form
\equ{
\frac{\partial A}{\partial s}^{1}_{full}=\left\{
\begin{array}{ll}
2\frac{A^{2}_{half}}{ds} & \textrm{:}\quad m=1 \\
0                        & \textrm{:}\quad m\ne1\\
\end{array}
\right.
}{equ:half_ds_ghost}
Except when the quanity contains a radial derivative, there the $A^{-1}$ is negative resulting in
\equ{
\frac{\partial A}{\partial s}^{1}_{full}=\left\{
\begin{array}{ll}
2\frac{A^{2}_{half}}{ds} & \textrm{:}\quad m=0 \\
0                        & \textrm{:}\quad m\ne0\\
\end{array}
\right.
}{equ:half_ds_ghost}

\subsection{Jacobian}
From equation \ref{equ:jacobian} and the fourier definition of the covariant basis vectors, the jacobian can be written in fourier terms.
Starting with the $\crossp{\co{u}}{\co{v}}$ term
\equ{
\crossp{\co{u}}{\co{v}}=\left|
\begin{array}{ccc}
\unitv{r} & \unitv{\phi} & \unitv{z} \\
\pd{R}{u} & 0            & \pd{Z}{u} \\
\pd{R}{v} & R            & \pd{Z}{v}
\end{array}
\right|=-R\pd{Z}{u}\unitv{r}+\brackets{\pd{Z}{u}\pd{R}{v}-\pd{R}{u}\pd{Z}{v}}\unitv{\phi}+R\pd{R}{u}\unitv{v}
}{equ:jacobian_cross}
Take the dot product with the $\co{s}$ term results in
\equ{
\dotp{\co{s}}{\crossp{\co{u}}{\co{v}}}=-R\pd{R}{s}\pd{Z}{u}+R\pd{Z}{s}\pd{R}{u}
}{equ:jacobian_full}
Since the Jacobian contains radial derivatives, it is most convenient to treat this a half grid quantity.
From the full to half grid and radial derivatives, equations \ref{equ:full_to_half} and \ref{equ:full_ds}, the half grid Jacobian can be written as.
\equ{
\mathcal{J}^{i}_{half}=\\
-\frac{\brackets{R^{i}_{full}+R^{i-1}_{full}}}{2}\cos\brackets{m_{R}u+n_{R}v}\\
\frac{\brackets{R^{i}_{full}-R^{i-1}_{full}}}{ds}\cos\brackets{m_{dR}u+n_{dR}v}\\
\frac{\brackets{Z^{i}_{full}+Z^{i-1}_{full}}}{2}m_{Z}\cos\brackets{m_{Z}u+n_{Z}v}\\
-\frac{\brackets{R^{i}_{full}+R^{i-1}_{full}}}{2}\cos\brackets{m_{R}u+n_{R}v}\\
\frac{\brackets{Z^{i}_{full}-Z^{i-1}_{full}}}{ds}\sin\brackets{m_{dZ}u+n_{dZ}v}\\
\frac{\brackets{R^{i}_{full}+R^{i-1}_{full}}}{2}m_{R}\sin\brackets{m_{R}u+n_{R}v}\\
\quad i=2\rightarrow ns
}{equ:jacobian_half}
The triple $\cos$ product reduces using the product rule to.
\equ{
\frac{1}{2}\cos\sqbrackets{\brackets{m_{R}-m_{dR}}u+\brackets{n_{R}-n_{dR}}v}\cos\brackets{m_{z}u+n_{z}v}\\
+\frac{1}{2}\cos\sqbrackets{\brackets{m_{R}+m_{dR}}u+\brackets{n_{R}+n_{dR}}v}\cos\brackets{m_{z}u+n_{z}v}
}{equ:jacobian_cos}
\equ{
\frac{1}{4}\cos\sqbrackets{\brackets{m_{R}-m_{dR}-m_{z}}u+\brackets{n_{R}-n_{dR}-n_{Z}}v}\\
+\frac{1}{4}\cos\sqbrackets{\brackets{m_{R}-m_{dR}+m_{z}}u+\brackets{n_{R}-n_{dR}+n_{z}}v}\\
+\frac{1}{4}\cos\sqbrackets{\brackets{m_{R}+m_{dR}-m_{z}}u+\brackets{n_{R}+n_{dR}-n_{z}}v}\\
+\frac{1}{4}\cos\sqbrackets{\brackets{m_{R}+m_{dR}+m_{z}}u+\brackets{n_{R}+n_{dR}+n_{z}}v}
}{equ:jacobian_cos2}
The $\cos\sin\sin$ product reduces to.
\equ{
\frac{1}{2}\sin\sqbrackets{\brackets{m_{R}+m_{dR}}u+\brackets{n_{R}+n_{dR}}v}\sin\brackets{m_{z}u+n_{z}v}\\
-\frac{1}{2}\sin\sqbrackets{\brackets{m_{R}-m_{dR}}u+\brackets{n_{R}-n_{dR}}v}\sin\brackets{m_{z}u+n_{z}v}
}{equ:jacobian_sin}
\equ{
\frac{1}{4}\cos\sqbrackets{\brackets{m_{R}+m_{dR}-m_{z}}u+\brackets{n_{R}+n_{dR}-n_{Z}}v}\\
-\frac{1}{4}\cos\sqbrackets{\brackets{m_{R}+m_{dR}+m_{z}}u+\brackets{n_{R}+n_{dR}+n_{z}}v}\\
-\frac{1}{4}\cos\sqbrackets{\brackets{m_{R}-m_{dR}-m_{z}}u+\brackets{n_{R}-n_{dR}-n_{z}}v}\\
+\frac{1}{4}\cos\sqbrackets{\brackets{m_{R}-m_{dR}+m_{z}}u+\brackets{n_{R}-n_{dR}+n_{z}}v}
}{equ:jacobian_sin2}

Putting equations \ref{equ:jacobian_cos2} and \ref{equ:jacobian_sin2} into \ref{equ:jacobian_half}.
\equ{
\mathcal{J}^{i}_{half}=\\
-\frac{R^{i}_{full}+R^{i-1}_{full}}{2}\frac{R^{i}_{full}-R^{i-1}_{full}}{ds}\frac{Z^{i}_{full}+Z^{i-1}_{full}}{2}m_{Z}\frac{1}{4}\cos\sqbrackets{\brackets{m_{R}-m_{dR}-m_{Z}}u+\brackets{n_{R}-n_{dR}-n_{Z}}v}\\
-\frac{R^{i}_{full}+R^{i-1}_{full}}{2}\frac{R^{i}_{full}-R^{i-1}_{full}}{ds}\frac{Z^{i}_{full}+Z^{i-1}_{full}}{2}m_{Z}\frac{1}{4}\cos\sqbrackets{\brackets{m_{R}-m_{dR}+m_{Z}}u+\brackets{n_{R}-n_{dR}+n_{z}}v}\\
-\frac{R^{i}_{full}+R^{i-1}_{full}}{2}\frac{R^{i}_{full}-R^{i-1}_{full}}{ds}\frac{Z^{i}_{full}+Z^{i-1}_{full}}{2}m_{Z}\frac{1}{4}\cos\sqbrackets{\brackets{m_{R}+m_{dR}-m_{Z}}u+\brackets{n_{R}+n_{dR}-n_{z}}v}\\
-\frac{R^{i}_{full}+R^{i-1}_{full}}{2}\frac{R^{i}_{full}-R^{i-1}_{full}}{ds}\frac{Z^{i}_{full}+Z^{i-1}_{full}}{2}m_{Z}\frac{1}{4}\cos\sqbrackets{\brackets{m_{R}+m_{dR}+m_{Z}}u+\brackets{n_{R}+n_{dR}+n_{z}}v}\\
-\frac{R^{i}_{full}+R^{i-1}_{full}}{2}\frac{Z^{i}_{full}-Z^{i-1}_{full}}{ds}\frac{R^{i}_{full}+R^{i-1}_{full}}{2}m_{R}\frac{1}{4}\cos\sqbrackets{\brackets{m_{R}+m_{dR}-m_{Z}}u+\brackets{n_{R}+n_{dR}-n_{Z}}v}\\
+\frac{R^{i}_{full}+R^{i-1}_{full}}{2}\frac{Z^{i}_{full}-Z^{i-1}_{full}}{ds}\frac{R^{i}_{full}+R^{i-1}_{full}}{2}m_{R}\frac{1}{4}\cos\sqbrackets{\brackets{m_{R}+m_{dR}+m_{Z}}u+\brackets{n_{R}+n_{dR}+n_{z}}v}\\
+\frac{R^{i}_{full}+R^{i-1}_{full}}{2}\frac{Z^{i}_{full}-Z^{i-1}_{full}}{ds}\frac{R^{i}_{full}+R^{i-1}_{full}}{2}m_{R}\frac{1}{4}\cos\sqbrackets{\brackets{m_{R}-m_{dR}-m_{Z}}u+\brackets{n_{R}-n_{dR}-n_{z}}v}\\
-\frac{R^{i}_{full}+R^{i-1}_{full}}{2}\frac{Z^{i}_{full}-Z^{i-1}_{full}}{ds}\frac{R^{i}_{full}+R^{i-1}_{full}}{2}m_{R}\frac{1}{4}\cos\sqbrackets{\brackets{m_{R}-m_{dR}+m_{Z}}u+\brackets{n_{R}-n_{dR}+n_{z}}v}\\
\quad i=2\rightarrow ns
}{equ:jacobian_half2}

When this is converted down to the $s=0$ point, only the $m=0$ terms are retained for $m_{R}$ and $m_{Z}$.
Since the $\co{s}$ terms is nonzero for $m=1$, the $m_{dR}=1$.
Since the Jacobian contains a radial derivative.

\subsection{$\divergence{\curl{\vec{A}}} = 0$ Half Grid}
\label{sec:div_half}
We need to check the boundary conditions to ensure that magnetic fields remain divergence free.
The deviation of the Curl operator in flux coordinates is defined as
\equ{
\curl{\vec{A}}=\frac{1}{\mathcal{J}}\sum_{v}\brackets{\frac{\partial A_{u}}{\partial s}-\frac{\partial A_{s}}{\partial u}}\co{v}\quad\textrm{for cyclic perumations of }s,u,v
}{equ:curl}
The divergence operator is
\equ{
\divergence{\vec{A}}=\frac{1}{\mathcal{J}}\frac{\partial}{\partial i}\mathcal{J}A^{i}\quad i\rightarrow s,u,v
}{equ:div}

Starting from a full grid quantity $\vec{A}$, the Curl results in a half grid quantity $\vec{B}=\curl{\vec{A}}$.
$A^{s}$ has $\sin$ parity while $A^{u}$ and $A^{v}$ have $\cos$ parity.
From the definition of the curl,
\equ{
\mathcal{J}B^{s}=\frac{\partial A_{v}}{\partial u}_{half} - \frac{\partial A_{u}}{\partial v}_{half}=-m\brackets{A_{v}}_{half}+nN\brackets{A_{u}}_{half}
}{equ:curl_b_s}
\equ{
\brackets{\mathcal{J}B^{s}}^{i}_{half}=-m\frac{\brackets{A_{v}}^{i}_{full}+\brackets{A_{v}}^{i-1}_{full}}{2}+nN\frac{\brackets{A_{u}}^{i}_{full}+\brackets{A_{u}}^{i-1}_{full}}{2}\quad i=2\rightarrow ns
}{equ:curl_b_s_i}
\equ{
\mathcal{J}B^{u}=\frac{\partial A_{s}}{\partial v}_{half}-\frac{\partial A_{v}}{\partial s}_{half}=nN\brackets{A_{s}}_{half}-\frac{\partial A_{v}}{\partial s}_{half}
}{equ:curl_b_u}
\equ{
\brackets{\mathcal{J}B^{u}}^{i}_{half}=
nN\frac{\brackets{A_{s}}^{i}_{full}+\brackets{A_{s}}^{i-1}_{full}}{2}-\frac{\brackets{A_{v}}^{i}_{full}-\brackets{A_{v}}^{i-1}_{full}}{ds}\quad i=2\rightarrow ns
}{equ:curl_b_u_i}
\equ{
\mathcal{J}B^{v}=\frac{\partial A_{u}}{\partial s}_{half}-\frac{\partial A_{s}}{\partial u}_{half}=\frac{\partial A_{u}}{\partial s}_{half}-m\brackets{A_{s}}_{half}
}{equ:curl_b_v}
\equ{
\brackets{\mathcal{J}B^{v}}^{i}_{half}=\frac{\brackets{A_{u}}^{i}_{full}-\brackets{A_{u}}^{i-1}_{full}}{ds}-m\frac{\brackets{A_{s}}^{i}_{full}+\brackets{A_{s}}^{i-1}_{full}}{2}\quad i=2\rightarrow ns
}{equ:curl_b_v_i}
Since $A_{s}$ has $\sin$ parity, and $A_{u}$ and $A_{v}$ have $\cos$ parity, $B^{s}$ will have $\sin$ parity, and $B^{u}$ and $B^{v}$ will have $\cos$ parity.

\subsubsection{Non-boundary Terms}
The non-boundary components of the divergence fall on the full grid.
\equ{
\brackets{\frac{\partial}{\partial s}\mathcal{J}B^{s}}^{i}_{full}=\frac{\brackets{\mathcal{J}B^{s}}^{i+1}_{half}-\brackets{\mathcal{J}B^{s}}^{i}_{half}}{ds}\quad i=2\rightarrow ns-1
}{equ:div_jb_s}
Using equation \ref{equ:curl_b_s_i} this becomes
\equ{
\brackets{\frac{\partial}{\partial s}\mathcal{J}B^{s}}^{i}_{full}=-m\frac{\brackets{A_{v}}^{i+1}_{full}+\brackets{A_{v}}^{i}_{full}}{2ds}+nN\frac{\brackets{A_{u}}^{i+1}_{full}+\brackets{A_{u}}^{i}_{full}}{2ds}\\
+m\frac{\brackets{A_{v}}^{i}_{full}+\brackets{A_{v}}^{i-1}_{full}}{2ds}-nN\frac{\brackets{A_{u}}^{i}_{full}+\brackets{A_{u}}^{i-1}_{full}}{2ds}\\i=2\rightarrow ns-1
}{equ:div_jb_s_1}
All $A^{i}$ terms cancel leaving
\equ{
\brackets{\frac{\partial}{\partial s}\mathcal{J}B^{s}}^{i}_{full}=-m\frac{\brackets{A_{v}}^{i+1}_{full}}{2ds}+nN\frac{\brackets{A_{u}}^{i+1}_{full}}{2ds}+m\frac{\brackets{A_{v}}^{i-1}_{full}}{2ds}-nN\frac{\brackets{A_{u}}^{i-1}_{full}}{2ds}\quad i=2\rightarrow ns-1
}{equ:div_jb_s_2}
The next term
\equ{
\brackets{\frac{\partial}{\partial u}\mathcal{J}B^{u}}^{i}_{full}=-m\brackets{\mathcal{J}B^{u}}^{i}_{full}=
-m\frac{\brackets{\mathcal{J}B^{u}}^{i+1}_{half} + \brackets{\mathcal{J}B^{u}}^{i}_{half}}{2}\quad i=2\rightarrow ns-1
}{equ:div_jb_u}
Using equation \ref{equ:curl_b_u_i} this becomes
\equ{
\brackets{\frac{\partial}{\partial u}\mathcal{J}B^{u}}^{i}_{full}=-mnN\frac{\brackets{A_{s}}^{i+1}_{full}+\brackets{A_{s}}^{i}_{full}}{4}+m\frac{\brackets{A_{v}}^{i+1}_{full}-\brackets{A_{v}}^{i}_{full}}{2ds}\\
-mnN\frac{\brackets{A_{s}}^{i}_{full}+\brackets{A_{s}}^{i-1}_{full}}{4}+m\frac{\brackets{A_{v}}^{i}_{full}-\brackets{A_{v}}^{i-1}_{full}}{2ds}\\
i=2\rightarrow ns-1
}{equ:div_jb_u_1}
All $A^{i}_{v}$ terms cancel leaving
\equ{
\brackets{\frac{\partial}{\partial u}\mathcal{J}B^{u}}^{i}_{full}=-mnN\frac{\brackets{A_{s}}^{i+1}_{full}+\brackets{A_{s}}^{i}_{full}}{4}+m\frac{\brackets{A_{v}}^{i+1}_{full}}{2ds}\\
-mnN\frac{\brackets{A_{s}}^{i}_{full}+\brackets{A_{s}}^{i-1}_{full}}{4}-m\frac{\brackets{A_{v}}^{i-1}_{full}}{2ds}\\
i=2\rightarrow ns-1
}{equ:div_jb_u_2}
The last term
\equ{
\brackets{\frac{\partial}{\partial v}\mathcal{J}B^{v}}^{i}_{full}=-nN\brackets{\mathcal{J}B^{v}}^{i}_{full}=
-nN\frac{\brackets{\mathcal{J}B^{v}}^{i+1}_{half} + \brackets{\mathcal{J}B^{v}}^{i}_{half}}{2}\quad i=2\rightarrow ns-1
}{equ:div_jb_v}
Using equation \ref{equ:curl_b_v_i} this becomes
\equ{
\brackets{\frac{\partial}{\partial v}\mathcal{J}B^{v}}^{i}_{full}=-nN\frac{\brackets{A_{u}}^{i+1}_{full}-\brackets{A_{u}}^{i}_{full}}{2ds}+mnN\frac{\brackets{A_{s}}^{i+1}_{full}+\brackets{A_{s}}^{i}_{full}}{4}\\
-nN\frac{\brackets{A_{u}}^{i}_{full}-\brackets{A_{u}}^{i-1}_{full}}{2ds}+mnN\frac{\brackets{A_{s}}^{i}_{full}+\brackets{A_{s}}^{i-1}_{full}}{4}\\
i=2\rightarrow ns-1
}{equ:div_jb_v_1}
All $A^{i}_u$ terms cancel leaving
\equ{
\brackets{\frac{\partial}{\partial v}\mathcal{J}B^{v}}^{i}_{full}=-nN\frac{\brackets{A_{u}}^{i+1}_{full}}{2ds}+mnN\frac{\brackets{A_{s}}^{i+1}_{full}+\brackets{A_{s}}^{i}_{full}}{4}\\
+nN\frac{\brackets{A_{u}}^{i-1}_{full}}{2ds}+mnN\frac{\brackets{A_{s}}^{i}_{full}+\brackets{A_{s}}^{i-1}_{full}}{4}\\
i=2\rightarrow ns-1
}{equ:div_jb_v_2}
Combining equations \ref{equ:div_jb_s_2}, \ref{equ:div_jb_u_2}, and \ref{equ:div_jb_v_2}
\equ{
\brackets{\frac{\partial}{\partial s}\mathcal{J}B^{s}}^{i}_{full}+\brackets{\frac{\partial}{\partial u}\mathcal{J}B^{u}}^{i}_{full}+\brackets{\frac{\partial}{\partial v}\mathcal{J}B^{v}}^{i}_{full}=\\
-m\frac{\brackets{A_{v}}^{i+1}_{full}}{2ds}+nN\frac{\brackets{A_{u}}^{i+1}_{full}}{2ds}+m\frac{\brackets{A_{v}}^{i-1}_{full}}{2ds}-nN\frac{\brackets{A_{u}}^{i-1}_{full}}{2ds}\\
-mnN\frac{\brackets{A_{s}}^{i+1}_{full}+\brackets{A_{s}}^{i}_{full}}{4}+m\frac{\brackets{A_{v}}^{i+1}_{full}}{2ds}-mnN\frac{\brackets{A_{s}}^{i}_{full}+\brackets{A_{s}}^{i-1}_{full}}{4}-m\frac{\brackets{A_{v}}^{i-1}_{full}}{2ds}\\
-nN\frac{\brackets{A_{u}}^{i+1}_{full}}{2ds}+mnN\frac{\brackets{A_{s}}^{i+1}_{full}+\brackets{A_{s}}^{i}_{full}}{4}+nN\frac{\brackets{A_{u}}^{i-1}_{full}}{2ds}+mnN\frac{\brackets{A_{s}}^{i}_{full}+\brackets{A_{s}}^{i-1}_{full}}{4}\\
i=2\rightarrow ns-1
}{equ:div_total}
Canceling out terms results in
\equ{
\brackets{\frac{\partial}{\partial s}\mathcal{J}B^{s}}^{i}_{full}+\brackets{\frac{\partial}{\partial u}\mathcal{J}B^{u}}^{i}_{full}+\brackets{\frac{\partial}{\partial v}\mathcal{J}B^{v}}^{i}_{full}=0\\
i=2\rightarrow ns-1
}{equ:div_total_1}

\subsubsection{$ns$ Terms}
The first $ns$ boundary component is
\equ{
\brackets{\frac{\partial}{\partial s}\mathcal{J}B^{s}}^{ns}_{full}=\frac{\brackets{\mathcal{J}B^{s}}^{ns}_{half}-\brackets{\mathcal{J}B^{s}}^{ns-1}_{half}}{ds}
}{equ:div_jb_s_ns}
Using equation \ref{equ:curl_b_s_i} this expands to
\equ{
\brackets{\frac{\partial}{\partial s}\mathcal{J}B^{s}}^{ns}_{full}=-m\frac{\brackets{A_{v}}^{ns}_{full}}{2ds}-m\frac{\brackets{A_{v}}^{ns-1}_{full}}{2ds}+nN\frac{\brackets{A_{u}}^{ns}_{full}}{2ds}+nN\frac{\brackets{A_{u}}^{ns-1}_{full}}{2ds}\\
+m\frac{\brackets{A_{v}}^{ns-1}_{full}}{2ds}+m\frac{\brackets{A_{v}}^{ns-2}_{full}}{2ds}-nN\frac{\brackets{A_{u}}^{ns-1}_{full}}{2ds}-nN\frac{\brackets{A_{u}}^{ns-2}_{full}}{2ds}
}{equ:div_jb_s_ns_1}
Which reduces to
\equ{
\brackets{\frac{\partial}{\partial s}\mathcal{J}B^{s}}^{ns}_{full}=-m\frac{\brackets{A_{v}}^{ns}_{full}}{2ds}+nN\frac{\brackets{A_{u}}^{ns}_{full}}{2ds}\\
+m\frac{\brackets{A_{v}}^{ns-2}_{full}}{2ds}-nN\frac{\brackets{A_{u}}^{ns-2}_{full}}{2ds}
}{equ:div_jb_s_ns_2}

The second $ns$ boundary component is
\equ{
\brackets{\frac{\partial}{\partial u}\mathcal{J}B^{u}}^{ns}_{full}=-m\brackets{\mathcal{J}B^{u}}^{ns}_{full}=-m\brackets{\mathcal{J}B^{u}}^{ns}_{half}
}{equ:div_jb_u_ns}
Using equation \ref{equ:curl_b_u_i} this expands to
\equ{
\brackets{\frac{\partial}{\partial u}\mathcal{J}B^{u}}^{ns}_{full}=-mnN\frac{\brackets{A_{s}}^{ns}_{full}}{2}-mnN\frac{\brackets{A_{s}}^{ns-1}_{full}}{2}+m\frac{\brackets{A_{v}}^{ns}_{full}}{ds}-m\frac{\brackets{A_{v}}^{ns-1}_{full}}{ds}
}{equ:div_jb_u_ns_1}
The third ns boundary component is
\equ{
\brackets{\frac{\partial}{\partial v}\mathcal{J}B^{v}}^{ns}_{full}=-nN\brackets{\mathcal{J}B^{v}}^{ns}_{full}=-nN\brackets{\mathcal{J}B^{v}}^{ns}_{half}
}{equ:div_jb_v_ns}
Using equation \ref{equ:curl_b_v_i} this expands to
\equ{
\brackets{\frac{\partial}{\partial v}\mathcal{J}B^{v}}^{ns}_{full}=-nN\frac{\brackets{A_{u}}^{ns}_{full}}{ds}+nN\frac{\brackets{A_{u}}^{ns-1}_{full}}{ds}+mnN\frac{\brackets{A_{s}}^{ns}_{full}}{2}+mnN\frac{\brackets{A_{s}}^{ns-1}_{full}}{2}
}{equ:div_jb_v_ns_1}

Canceling out common terms results in
\equ{
\brackets{\frac{\partial}{\partial s}\mathcal{J}B^{s}}^{ns}_{full}+\brackets{\frac{\partial}{\partial u}\mathcal{J}B^{u}}^{ns}_{full}+\brackets{\frac{\partial}{\partial v}\mathcal{J}B^{v}}^{ns}_{full}=0
}{equ:div_total_ns_1}

\subsubsection{Axis Terms}
\label{sub:sec:div_axis}
Note these are no longer vector quantities so we use the \ref{equ:half_to_full_1} and \ref{equ:half_ds_ghost} conditions here.
The first axis boundary component is
\equ{
\brackets{\frac{\partial}{\partial s}\mathcal{J}B^{s}}^{1}_{full}=\frac{2\brackets{\mathcal{J}B^{s}}^{2}_{half}}{ds}\quad\textrm{:}\quad m=1
}{equ:div_jb_s_ghost}
From equation \ref{equ:curl_b_s_i} this expands to
\equ{
\brackets{\frac{\partial}{\partial s}\mathcal{J}B^{s}}^{1}_{full}=-m\frac{\brackets{A_{v}}^{2}_{full}}{ds}-m\frac{\brackets{A_{v}}^{1}_{full}}{ds}+nN\frac{\brackets{A_{u}}^{2}_{full}}{ds}+nN\frac{\brackets{A_{u}}^{1}_{full}}{ds}\quad\textrm{:}\quad m=1
}{equ:div_jb_s_ghost_2}
From conditions \ref{equ:asubu} and \ref{equ:asubv}, at the origin $A_{u}=0$, and $A_v=0$ for all $m>0$ causing this term to reduce to
\equ{
\brackets{\frac{\partial}{\partial s}\mathcal{J}B^{s}}^{1}_{full}=-m\frac{\brackets{A_{v}}^{2}_{full}}{ds}+nN\frac{\brackets{A_{u}}^{2}_{full}}{ds}\quad\textrm{:}\quad m=1
}{equ:div_jb_s_ghost_3}

The second axis boundary component is
\equ{
\brackets{\frac{\partial}{\partial u}\mathcal{J}B^{u}}^{1}_{full}=-m\brackets{\mathcal{J}B^{u}}^{1}_{full}=-m\brackets{\mathcal{J}B^{u}}^{2}_{half}
}{equ:div_jb_u_ghost}
From equation \ref{equ:covariant_boundary_u}, only the $m=1$ terms of $JB^{u}$ should remain non zero.
Since the curl term contains a radial derivative, the condition \ref{equ:half_to_full_2} is used here.
The conversion from the half grid retains only $m=1$ terms consistent with the condition \ref{equ:covariant_boundary_u}.
Putting equation \ref{equ:curl_b_u_i}
\equ{
\brackets{\frac{\partial}{\partial u}\mathcal{J}B^{u}}^{1}_{full}=-mnN\frac{\brackets{A_{s}}^{2}_{full}}{2}-mnN\frac{\brackets{A_{s}}^{1}_{full}}{2}+m\frac{\brackets{A_{v}}^{2}_{full}}{ds}-m\frac{\brackets{A_{v}}^{1}_{full}}{ds}\quad\textrm{:}\quad m=1
}{equ:div_jb_u_ghost_2}
$A_{v}=0$ for $m=1$ terms based on the covariant basis vectors \ref{equ:covariant_boundary_v}.
\equ{
\brackets{\frac{\partial}{\partial u}\mathcal{J}B^{u}}^{1}_{full}=-mnN\frac{\brackets{A_{s}}^{2}_{full}}{2}-mnN\frac{\brackets{A_{s}}^{1}_{full}}{2}+m\frac{\brackets{A_{v}}^{2}_{full}}{ds}\quad\textrm{:}\quad m=1
}{equ:div_jb_u_ghost_3}

The last axis boundary component is
\equ{
\brackets{\frac{\partial}{\partial v}\mathcal{J}B^{v}}^{1}_{full}=-nN\brackets{\mathcal{J}B^{v}}^{1}_{full}=-nN\brackets{\mathcal{J}B^{v}}^{2}_{half}
}{equ:div_jb_v_ghost}
Since the curl term contains a radial derivative, the condition \ref{equ:half_to_full_2} is used here retaining only $m=1$ terms.
From equation \ref{equ:curl_b_v_i} this expands to
\equ{
\brackets{\frac{\partial}{\partial v}\mathcal{J}B^{v}}^{1}_{full}=-nN\frac{\brackets{A_{u}}^{2}_{full}}{ds}+nN\frac{\brackets{A_{u}}^{1}_{full}}{ds}+mnN\frac{\brackets{A_{s}}^{2}_{full}}{2}+mnN\frac{\brackets{A_{s}}^{1}_{full}}{2}\quad\textrm{:}\quad m=1
}{equ:div_jb_v_ghost_2}
$A_{v}=0$ for $m=1$ terms based on the covariant basis vectors \ref{equ:covariant_boundary_u}.
\equ{
\brackets{\frac{\partial}{\partial v}\mathcal{J}B^{v}}^{1}_{full}=-nN\frac{\brackets{A_{u}}^{2}_{full}}{ds}+mnN\frac{\brackets{A_{s}}^{2}_{full}}{2}+mnN\frac{\brackets{A_{s}}^{1}_{full}}{2}\quad\textrm{:}\quad m=1
}{equ:div_jb_v_ghost_3}

Combining equations \ref{equ:div_jb_s_ghost_3}, \ref{equ:div_jb_u_ghost_3}, and \ref{equ:div_jb_v_ghost_3} together,
\equ{
\brackets{\frac{\partial}{\partial s}\mathcal{J}B^{s}}^{1}_{full}+\brackets{\frac{\partial}{\partial u}\mathcal{J}B^{u}}^{1}_{full}+\brackets{\frac{\partial}{\partial v}\mathcal{J}B^{v}}^{1}_{full}=\\
-m\frac{\brackets{A_{v}}^{2}_{full}}{ds}+nN\frac{\brackets{A_{u}}^{2}_{full}}{ds}\\
-mnN\frac{\brackets{A_{s}}^{2}_{full}}{2}-mnN\frac{\brackets{A_{s}}^{1}_{full}}{2}+m\frac{\brackets{A_{v}}^{2}_{full}}{ds}\\
-nN\frac{\brackets{A_{u}}^{2}_{full}}{ds}+mnN\frac{\brackets{A_{s}}^{2}_{full}}{2}+mnN\frac{\brackets{A_{s}}^{1}_{full}}{2}\quad\textrm{:}\quad m=1
}{equ:div_total_1_1}
Canceling out terms results in 
\equ{
\brackets{\frac{\partial}{\partial s}\mathcal{J}B^{s}}^{1}_{full}+\brackets{\frac{\partial}{\partial u}\mathcal{J}B^{u}}^{1}_{full}+\brackets{\frac{\partial}{\partial v}\mathcal{J}B^{v}}^{1}_{full}=0
}{equ:div_total_1_2}

\subsection{$\divergence{\curl{\vec{A}}} = 0$ Full Grid}
Starting from the half grid quantity $\vec{B}$, the Curl results in a full grid quantity $\vec{J}=\curl{\vec{B}}$.
$B^{s}$ has sin parity while $B^{u}$ and $B^{v}$ have cos parity. From the definition of the curl,
\equ{
\mathcal{J}B^{s}=\frac{\partial A_{v}}{\partial u}_{full} - \frac{\partial A_{u}}{\partial v}_{full}=-m\brackets{A_{v}}_{full}+nN\brackets{A_{u}}_{full}
}{equ:curl_j_s}
\equ{
\mathcal{J}A^{u}=\frac{\partial A_{s}}{\partial v}_{full}-\frac{\partial A_{v}}{\partial s}_{half}=nN\brackets{A_{s}}_{full}-\frac{\partial A_{v}}{\partial s}_{full}
}{equ:curl_j_u}
\equ{
\mathcal{J}A^{v}=\frac{\partial A_{u}}{\partial s}_{full}-\frac{\partial A_{s}}{\partial u}_{full}=\frac{\partial A_{u}}{\partial s}_{full}-m\brackets{A_{s}}_{full}
}{equ:curl_j_v}

\subsubsection{Non-Boundary Terms}
For the non boundary term, equations \ref{equ:curl_j_s}, \ref{equ:curl_j_u} and \ref{equ:curl_j_v} become
\equ{
\brackets{\mathcal{J}J^{s}}^{i}_{full}=-m\frac{\brackets{B_{v}}^{i+1}_{half}+\brackets{B_{v}}^{i}_{half}}{2}+nN\frac{\brackets{B_{u}}^{i+1}_{half}+\brackets{B_{u}}^{i}_{half}}{2}\quad i=2\rightarrow ns-1
}{equ:curl_j_s_i}
\equ{
\brackets{\mathcal{J}J^{u}}^{i}_{full}=
nN\frac{\brackets{B_{s}}^{i+1}_{half}+\brackets{B_{s}}^{i}_{half}}{2}-\frac{\brackets{B_{v}}^{i+1}_{half}-\brackets{B_{v}}^{i}_{half}}{ds}\quad i=2\rightarrow ns-1
}{equ:curl_j_u_i}
\equ{
\brackets{\mathcal{J}J^{v}}^{i}_{full}=\frac{\brackets{B_{u}}^{i+1}_{half}-\brackets{B_{u}}^{i}_{half}}{ds}-m\frac{\brackets{B_{s}}^{i+1}_{half}+\brackets{B_{s}}^{i}_{half}}{2}\quad i=2\rightarrow ns-1
}{equ:curl_j_v_i}
Since $B_{s}$ has $\sin$ parity, and $B_{u}$ and $B_{v}$ have $\cos$ parity, $J^{s}$ will have $\sin$ parity, and $J^{u}$ and $J^{v}$ will have $\cos$ parity.

The non-boundary components of the divergence fall on the half grid.
\equ{
\brackets{\frac{\partial}{\partial s}\mathcal{J}J^{s}}^{i}_{half}=\frac{\brackets{\mathcal{J}J^{s}}^{i}_{full}-\brackets{\mathcal{J}J^{s}}^{i-1}_{full}}{ds}\quad i=3\rightarrow ns-1
}{equ:div_jj_s}
Using equation \ref{equ:curl_j_s_i} this becomes
\equ{
\brackets{\frac{\partial}{\partial s}\mathcal{J}J^{s}}^{i}_{half}=-m\frac{\brackets{B_{v}}^{i+1}_{half}+\brackets{B_{v}}^{i}_{half}}{2ds}+nN\frac{\brackets{B_{u}}^{i+1}_{half}+\brackets{B_{u}}^{i}_{half}}{2ds}\\
+m\frac{\brackets{B_{v}}^{i}_{half}+\brackets{B_{v}}^{i-1}_{half}}{2ds}-nN\frac{\brackets{B_{u}}^{i}_{half}+\brackets{B_{u}}^{i-1}_{half}}{2ds}\\i=3\rightarrow ns-1
}{equ:div_jj_s_1}
All $B^{i}$ terms cancel leaving
\equ{
\brackets{\frac{\partial}{\partial s}\mathcal{J}J^{s}}^{i}_{half}=-m\frac{\brackets{B_{v}}^{i+1}_{half}}{2ds}+nN\frac{\brackets{B_{u}}^{i+1}_{half}}{2ds}+m\frac{\brackets{B_{v}}^{i-1}_{half}}{2ds}-nN\frac{\brackets{B_{u}}^{i-1}_{half}}{2ds}\quad i=3\rightarrow ns-1
}{equ:div_jj_s_2}
The next term
\equ{
\brackets{\frac{\partial}{\partial u}\mathcal{J}J^{u}}^{i}_{half}=-m\brackets{\mathcal{J}J^{u}}^{i}_{half}=
-m\frac{\brackets{\mathcal{J}J^{u}}^{i+1}_{full} + \brackets{\mathcal{J}J^{u}}^{i}_{full}}{2}\quad i=3\rightarrow ns-1
}{equ:div_jj_u}
Using equation \ref{equ:curl_j_u_i} this becomes
\equ{
\brackets{\frac{\partial}{\partial u}\mathcal{J}J^{u}}^{i}_{half}=-mnN\frac{\brackets{B_{s}}^{i+1}_{half}+\brackets{B_{s}}^{i}_{half}}{4}+m\frac{\brackets{B_{v}}^{i+1}_{half}-\brackets{B_{v}}^{i}_{half}}{2ds}\\
-mnN\frac{\brackets{B_{s}}^{i}_{half}+\brackets{B_{s}}^{i-1}_{half}}{4}+m\frac{\brackets{B_{v}}^{i}_{half}-\brackets{B_{v}}^{i-1}_{half}}{2ds}\\
i=3\rightarrow ns-1
}{equ:div_jj_u_1}
All $B^{i}_{v}$ terms cancel leaving
\equ{
\brackets{\frac{\partial}{\partial u}\mathcal{J}J^{u}}^{i}_{half}=-mnN\frac{\brackets{B_{s}}^{i+1}_{half}+\brackets{B_{s}}^{i}_{half}}{4}+m\frac{\brackets{B_{v}}^{i+1}_{half}}{2ds}\\
-mnN\frac{\brackets{B_{s}}^{i}_{half}+\brackets{B_{s}}^{i-1}_{half}}{4}-m\frac{\brackets{B_{v}}^{i-1}_{half}}{2ds}\\
i=3\rightarrow ns-1
}{equ:div_jj_u_2}
The last term
\equ{
\brackets{\frac{\partial}{\partial v}\mathcal{J}J^{v}}^{i}_{half}=-nN\brackets{\mathcal{J}J^{v}}^{i}_{half}=
-nN\frac{\brackets{\mathcal{J}J^{v}}^{i+1}_{full} + \brackets{\mathcal{J}J^{v}}^{i}_{full}}{2}\quad i=3\rightarrow ns-1
}{equ:div_jj_v}
Using equation \ref{equ:curl_j_v_i} this becomes
\equ{
\brackets{\frac{\partial}{\partial v}\mathcal{J}J^{v}}^{i}_{half}=-nN\frac{\brackets{B_{u}}^{i+1}_{half}-\brackets{B_{u}}^{i}_{half}}{2ds}+mnN\frac{\brackets{B_{s}}^{i+1}_{half}+\brackets{B_{s}}^{i}_{half}}{4}\\
-nN\frac{\brackets{B_{u}}^{i}_{half}-\brackets{B_{u}}^{i-1}_{half}}{2ds}+mnN\frac{\brackets{B_{s}}^{i}_{half}+\brackets{B_{s}}^{i-1}_{half}}{4}\\
i=3\rightarrow ns-1
}{equ:div_jj_v_1}
All $B^{i}_u$ terms cancel leaving
\equ{
\brackets{\frac{\partial}{\partial v}\mathcal{J}J^{v}}^{i}_{half}=-nN\frac{\brackets{B_{u}}^{i+1}_{half}}{2ds}+mnN\frac{\brackets{B_{s}}^{i+1}_{half}+\brackets{B_{s}}^{i}_{half}}{4}\\
+nN\frac{\brackets{B_{u}}^{i-1}_{half}}{2ds}+mnN\frac{\brackets{B_{s}}^{i}_{half}+\brackets{B_{s}}^{i-1}_{half}}{4}\\
i=3\rightarrow ns-1
}{equ:div_jj_v_2}
Combining equations \ref{equ:div_jj_s_2}, \ref{equ:div_jj_u_2}, and \ref{equ:div_jj_v_2}
\equ{
\brackets{\frac{\partial}{\partial s}\mathcal{J}J^{s}}^{i}_{half}+\brackets{\frac{\partial}{\partial u}\mathcal{J}J^{u}}^{i}_{half}+\brackets{\frac{\partial}{\partial v}\mathcal{J}J^{v}}^{i}_{half}=\\
-m\frac{\brackets{B_{v}}^{i+1}_{half}}{2ds}+nN\frac{\brackets{B_{u}}^{i+1}_{half}}{2ds}+m\frac{\brackets{B_{v}}^{i-1}_{half}}{2ds}-nN\frac{\brackets{B_{u}}^{i-1}_{half}}{2ds}\\
-mnN\frac{\brackets{B_{s}}^{i+1}_{half}+\brackets{B_{s}}^{i}_{half}}{4}+m\frac{\brackets{B_{v}}^{i+1}_{half}}{2ds}-mnN\frac{\brackets{B_{s}}^{i}_{half}+\brackets{B_{s}}^{i-1}_{half}}{4}-m\frac{\brackets{B_{v}}^{i-1}_{half}}{2ds}\\
-nN\frac{\brackets{B_{u}}^{i+1}_{half}}{2ds}+mnN\frac{\brackets{B_{s}}^{i+1}_{half}+\brackets{B_{s}}^{i}_{half}}{4}+nN\frac{\brackets{B_{u}}^{i-1}_{half}}{2ds}+mnN\frac{\brackets{B_{s}}^{i}_{half}+\brackets{B_{s}}^{i-1}_{half}}{4}\\
i=3\rightarrow ns-1
}{equ:div_j_total}
Canceling out terms results in
\equ{
\brackets{\frac{\partial}{\partial s}\mathcal{J}J^{s}}^{i}_{half}+\brackets{\frac{\partial}{\partial u}\mathcal{J}J^{u}}^{i}_{half}+\brackets{\frac{\partial}{\partial v}\mathcal{J}J^{v}}^{i}_{half}=0\\
i=3\rightarrow ns-1
}{equ:div_j_total_1}

\subsubsection{Near $ns$ Terms}
The first near $ns$ boundary component is
\equ{
\brackets{\frac{\partial}{\partial s}\mathcal{J}J^{s}}^{ns}_{half}=\frac{\brackets{\mathcal{J}J^{s}}^{ns}_{full}-\brackets{\mathcal{J}B^{s}}^{ns-1}_{full}}{ds}
}{equ:div_jj_s_ns}
The second near $ns$ boundary component is
\equ{
\brackets{\frac{\partial}{\partial u}\mathcal{J}B^{u}}^{ns}_{half}=-m\brackets{\mathcal{J}B^{u}}^{ns}_{half}=-m\frac{\brackets{\mathcal{J}B^{u}}^{ns}_{full}+\brackets{\mathcal{J}B^{u}}^{ns-1}_{full}}{2}
}{equ:div_jj_u_ns}
The third ns boundary component is
\equ{
\brackets{\frac{\partial}{\partial v}\mathcal{J}B^{v}}^{ns}_{full}=-nN\brackets{\mathcal{J}B^{v}}^{ns}_{full}=-nN\brackets{\mathcal{J}B^{v}}^{ns}_{full}
}{equ:div_jjM_v_ns}
Using equation \ref{equ:curl_b_v_i} this expands to
\equ{
\brackets{\frac{\partial}{\partial v}\mathcal{J}B^{v}}^{ns}_{full}=-nN\frac{\brackets{A_{u}}^{ns}_{full}}{ds}+nN\frac{\brackets{A_{u}}^{ns-1}_{full}}{ds}+mnN\frac{\brackets{A_{s}}^{ns}_{full}}{2}+mnN\frac{\brackets{A_{s}}^{ns-1}_{full}}{2}
}{equ:div_jj_v_ns_1}

\section{Perturbation}
\label{sec:perturbation}
To compute variations of equilibrium, SIESTA applies a displacement $\xi=v\Delta t$.
The variation of the magnetic field comes from the Faraday's law.
\equ{
\pd{\vec{B}}{t}=-\curl{\vec{E}}
}{equ:faraday}
The electric field is obtained from the magnetic induction
\equ{
\vec{E}=\crossp{\vec{v}}{\vec{B}}
}{equ:e_field}
The displacement $\xi$ contains an implicit $\mathcal{J}$ term.
When the taking the curl in real space ends up with electric field components without a $\mathcal{J}$ term removed.
The electric field is a full grid quantity.
To ensure that the divergence of the electric field perturbation remains divergence free, we apply the conditions from section \ref{sec:div_half}.

Additionally, the electric field using a covariant basis vector representation.
So all the same conditions that were applied to the vector potential in section \ref{sec:div_half} apply here as well.
\equ{
E^{full}_{s}\brackets{s=0}=0\quad m \ne 1
}{equ:covariant_e_boundary_s}
\equ{
E^{full}_{u}\brackets{s=0}=0
}{equ:covariant_e_boundary_u}
\equ{
E^{full}_{v}\brackets{s=0}=0\quad m \ne 0
}{equ:covariant_e_boundary_v}
The coupling conditions from section \ref{sub:sec:div_axis}.
\equ{
m\brackets{E_v}^{2}_{full}=nN\brackets{E_u}^{2}_{full}\quad m = 1
}{equ:covariant_e_s2_m1_couple}
\equ{
\brackets{E_u}^{2}_{full}=0\quad m=0
}{equ:covariant_eu_s2_m0}

\end{document}
